%!TEX root =../main/main.tex 
%Author: Eric Drechsler
\chapter[Statistics]{Statistics}
%+++++++++++++++++++++++++++++++++++
\label{ch:statistics}

\paragraph{Statistical Inference}
is the process of finding properties of a dataset by performing data analysis 
and understanding the characteristics of the underlying probability distribution 
the data is following.

\paragraph{Bayesian Inference} is a form of statistical inference. Properties are deducted from a dataset. 
Update the model as more information becomes available following Bayes Theorem.

$$
P(\theta|x)=\frac{P(x|\theta)P(\theta)}{P(x)}    
$$

Terminology:
$P(\theta)$ is the \textbf{Prior}. Represents our "beliefs" of the true value of the model parameters.
$P(\theta|x)$ is the \textbf{Posterior}. It represents the beliefs after having evaluated the data.
$P(x|\theta)$ is the \textbf{Likelihood}. How likely does our model reflect the data?

Therefore we can calculate the posterior distribution using our prior beliefs and updating the likelihood. 




\section{Bump hunts}

\subsection{(Un-) Conditional Fit}
An uncoditional macimum likelihood estimator (MLE) provides the best fit results
for a given set of parameters, including the free floating Parameters of
interest (POI). Concrete example: all nuissance parameters and signal strength are
floating, unconditional MLE is best result of adjusting model to data.
A conditional MLE is the best fit result when a POI is fixed to a certain value.
In the example, the siugnal strength is fixed at a value and NPs are estimated
for the best fit result.


