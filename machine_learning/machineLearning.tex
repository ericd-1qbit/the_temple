%!TEX root =../main/main.tex Author: Eric Drechsler
\chapter[Machine Learning]{Machine Learning}
%+++++++++++++++++++++++++++++++++++
\label{ch:ml}


%%%%%%%%%%%%%%%%%%%%%%%%%%%%%%%%%%%%%%%%%%%%%%%%%%%%%%%%%%%%%%%%%%%%
%<o><o><o><o><o><o><o><o><o><o><o><o><o><o><o><o><o><o><o><o><o><o>%
%%%%%%%%%%%%%%%%%%%%%%%%%%%%%%%%%%%%%%%%%%%%%%%%%%%%%%%%%%%%%%%%%%%%
\section{General}

Machine Learning can be separated into four areas of concern:
\paragraph{Classification} tasks are mapping input data onto a discrete set of
classes by learning a function $$ h:\mathbb{R}^m\rightarrow{1,2,...,t}
$$
Canonical example is to classify dog or cat in picture.

\paragraph{Regression} tasks  predict a value for given input by learning a function $$
h:\mathbb{R}^m\rightarrow\mathbb{R}
$$
Canonical example is the cost of a house given various external parameters (size, location, etc).

\paragraph{Anomaly detection} is for identifying deviations from the bulk behaviour of data. 
Canonical example is the detection of credit card fraud.
 
\paragraph{Denoising} tasks take a noisy input $\tilde{x}\in\mathbb{R}^m$ and map it to noise-free data $x\in\mathbb{R}^m$ by learning the underlying features.

\subsection{Formal Definition}
A formal definition given by Mitchel (1997) [56] 
\begin{quote}
  A computer program is said to learn from experience E with respect to some class of tasks T and performance measure P, 
  if the performance at tasks in T , as measured by P, improves with E.
\end{quote}

\subsection{Types of ML}
One can distinguish three different types of ML approaches, defined by the kind of input
provided to the algorithm.
\paragraph{Supervised} Learning tasks learn from labelled examples, i.e. a training dataset with 
data points $x$ and labels $y$.
\paragraph{Unsupervised} Learning tasks try to learn the underlying structure in a training data set without 
being provided specific labelling. The boundaries to supervised learning are blurry, one can transform 
an unsupervised task into a supervised one. One could marginalise over data features and do a supervised 
learning for the given feature. %TODO
$$
P(x)=\prod_{i=1}^mP(x_i|x_1,\ldots,x_{i-1})
$$

\paragraph{Reinforcement} learning does not rely on any dataset. The algorithm generates it's own data and 
updates the generative process according to a rules/reward functions.

\subsection{Important Definitions}
\begin{table}
  \begin{tabular}{r|l}
  Phrase & Explanation\\\hline
  Generalisation & Ability of an algorithm to perform on an unseen dataset.\\
  Features & Variables describing certain characteristics of the input data\\
  Feature Vector & One input data point instance\\
  train error & The error of the algorithm estimated on the training dataset.\\
  test error & The error of the algorithm estimated on the testing dataset.\\
  \end{tabular}
\end{table}




\subsection{Hands-On Material Collection}
Google colab-notebooks:\\
\url{https://drive.google.com/drive/folders/1ZXlzYVJnBa_QTS4uAa6JGO-KFzR9HQXk}

\begin{itemize}
  \item Which Considerations to take before translating your problem to ML?
  \begin{itemize}
    \item Does my task map to any ML paradigm?
	\item Can I write down my model analytically?
	\item What's the dimensionality of my data?
	\item data sample sizes?
  \end{itemize}
\end{itemize}

%%%%%%%%%%%%%%%%%%%%%%%%%%%%%%%%%%%%%%%%%%%%%%%%%%%%%%%%%%%%%%%%%%%%
\subsection{Unsorted Notes}
\paragraph{Random}
\begin{itemize}
  \item GAN - generate Ising model
  \item translate to QA??
  \item what's a feedforward NN backpropagation?
  \item I. Goodfellow has a great many taxonomies
  \item automatic differentiation - computing derivatives with se compuhda
  \item Recurrence: The output of a layer goes back to input.
  \item Hough transform: Image analysis technique which isolates features of a particular shape in an image.
  \item Dropout of 0.2 means 20\% of nodes in a layer are deactivated randomly
\end{itemize}




\paragraph{Weight Decay} is a form of regularisation of the model complexity.
Prevents over or underfitting. It is a penalty on the loss function for model
complexity. One could add all weights (in quadrature, since pos+neg) of a model to the loss function, resulting
in a potentially huge loss. Instead multiply the weights with a small factor:
the weight decay. But then during gradient descent, another term emerges for the
weight update, which depends on this small factor - that's why it's called
weight decay.\\
In math terms, our original loss $MSE(\hat{y},y)$ is updated with the weight
decay factor $w_d$ time the sum of all weights in quadrature in our model.
$$
loss=MSE(\hat{y},y)+w_d*\sum_i{w_i^2}
$$
For gradient descent with learning rate $\gamma$, the updated weight looks like this:
$$
w(t+1)=w(t)-\gamma+w_d*\frac{\mathrm{d}loss}{\mathrm{d}w}
$$
The derivative has an additional term wrt to the original loss:
$$
\frac{\mathrm{d}loss}{\mathrm{d}w}= \frac{\mathrm{d}MSE}{\mathrm{d}w}+2w_d\sum_i{w_i^2}
$$
and thereby contributes to the update of the weights, aka decay.

\subsection{Reading arXiv:1704.02685v2 - DeepLift - Learning Important Features through propagation of activation differences}
\subsubsection{Summary}
\begin{itemize}
  \item provide interpretability of NN output
  \item associate output to input features by backproping through all neurons 
  \item contribution scores of each neuron by comparing activation of n to a reference 
\end{itemize}
\subsubsection{Previous Methods}
\begin{itemize}
  \item Perturbation-based fwd prop approach: Observe impact on output after perturbing input or neurons.
 Problem here is that once input contribution has saturated the impact on output, impact of input on importance not attributed any further.
  Also computationally expensive due to additional fwd prop. 
 \item Guided backpropagation - Combination of Gradients and deconvolutional networks. Saliency Maps
 \item Gradient x Input - elementwise product. layerwise relevance propagation
 \item integrated gradients
 \item grad/guided CAM
\end{itemize}

\subsubsection{DeepLIFT}
Compute a contribution score for a given neuron $x$ to the output $t$.

\begin{tabular}{2}
  target neuron & t\\
  neurons or inputs & x\\
  reference value for smth & smth$^0$\\
  contribution score & $C_{\Delta x,\Delta t}$\\
  multiplier & smth$^0$\\
   & smth$^0$\\
\end{tabular}

\paragraph{Summation-To-Delta}
The contribution score is defined such that a summation over all changes from ref 
in a given layer of n neurons results is the change from ref in the target neuron.

\paragraph{Chain Rule for Multipliers}
Multipliers qunatify the contribution of the change in neuron x to the target change t. 
Similar to partial derivatives. For an additional layer, the Multipliers 
propagate from one layer to the next. Like chain rule - allows to apply backprop.

\paragraph{Defining a reference}
Problem dependent. Differences measured against what? 

\paragraph{Separating Positive And Negative contributions}

\paragraph{Contribution Score Assignment rules}
- Linear 
- Rescale
- RevealCancel

\paragraph{Target Layer Choice and Softmax}


\paragraph{Claimed Advantages}
\begin{itemize}
  \item zero gradients still allow for propagation of importance score
  \item avoid discontinuity artifacts in gradient calculations
  \item positive and negative contributions to importance separated - reveal dependencies missed by other methods 
\end{itemize}



\paragraph{Reading 1708.02949v3}
\begin{itemize}
  \item adversial approaches to mitigate mis-modelling effects - decreasing algorithmic performance
  \item 
  \item 
\end{itemize}


- ANN architecture (weights)
Was confused about the return value of keras getweights(). 
NN architecture: 784 input nodes, 128 nodes hidden, 32 nodes latent, 128 nodes
hidden, 784 output
weight returns arrays as follows: 
784x128 (weights input to each neuron in hidden)
128x1 (the WHOLE SECOND layer has 1 bias)
128x32 (weights hidden to latent)
32x1 (latent layer has 1 bias)
32x128 (weights second hidden layer)
128x1 (bias second hidden layer)
128x784 (weights output layer)
784x1 (ouptu layer bias)

input layer has no bias!

\paragraph{Reading 1711.09059v1 - Long Short-Term Memory (LSTM) networks with jet constituents for boosted top tagging at the LHC}
\begin{itemize}
  \item jet identification - deep learning approaches: calorimeter activation as image, jet constiuents to fully connected network
  \item recurrent NN - Long Short-Term Memory (LSTM) network (what's the relation?)
  \item Keras and Theano to train
  \item NN parameters:
	\begin{itemize}
	  \item network depth (=number of layers?)
	  \item number of nodes/layer
	  \item ReLU activation for hidden layers
	  \item sigmoid activation for output node
	  \item Adam optimiser with 40 epochs
	  \item early stopping criterion - patience parameter of 5 epochs on the loss in validation set
	  \item performance quantified by lowest binary cross entropy loss
	  \item value here: 4 hidden layers, 300-6 nodes
	\end{itemize}
  \item validating NN performance
	\begin{itemize}
	  \item ROC curve (bg rejection vs sig efficiency)
	  \item efficiency vs pt
	  \item scores train vs test (
	\end{itemize}

\end{itemize}


\subsection{Linear Regression}

\paragraph{Regression}
\begin{itemize}
  \item modelling target value given independent (from each other) predictors
  \item used in forecasting/cause and effect analyses between variables
  \item find exact relationship between independent variable set x and dependent variable y
\end{itemize}
\paragraph{Linear Regression}
\begin{itemize}
  \item relationship in question is linear
  \item relies on \textbf{cost function}
  \item measures difference between prediction and actual value y
  \item goal: find minimum of this cost function
  \item minimisation procedure: \emph{gradient descent} - update parameters iteratively
  \item stepwise approach minimum - number of steps corresponds to learning rate
  \item using partial derivatives to calculate updated parameters, eg. for parameters $a,b$ and learning rate $\alpha$:
\end{itemize}
\begin{align}
  a&=a-\alpha\frac{\delta J}{\delta a}\\
  b&=b-\alpha\frac{\delta J}{\delta b}\\
\end{align}
\paragraph{Logistics Regression}
\begin{itemize}
  \item relationship in question is linear
  \item relies on \textbf{cost function}
  \item measures difference between prediction and actual value y
  \item goal: find minimum of this cost function
  \item minimisation procedure: \emph{gradient descent} - update parameters iteratively
  \item stepwise approach minimum - number of steps corresponds to learning rate
  \item using partial derivatives to calculate updated parameters, eg. for parameters $a,b$ and learning rate $\alpha$:
\end{itemize}
\begin{align}
  a&=a-\alpha\frac{\delta J}{\delta a}\\
  b&=b-\alpha\frac{\delta J}{\delta b}\\
\end{align}


\subsection{PCA - Principal Component Analysis}

%https://towardsdatascience.com/a-one-stop-shop-for-principal-component-analysis-5582fb7e0a9c
\paragraph{Use Case:}

\begin{itemize}
  \item reduction of number of variables, without knowing which variables to completely remove
  \item ensure independence of all variables
  \item do not care about too much about interpretability of variables
\end{itemize}

\paragraph{Makes use of:}
\begin{itemize}
  \item measure of how each variable is associated with one another. (Covariance matrix.)
  \item directions in which our data are dispersed. (Eigenvectors.)
  \item relative importance of these different directions. (Eigenvalues.)
  \item PCA combines our predictors and allows us to drop the eigenvectors that are relatively unimportant.
\end{itemize}


\paragraph{Notes:}
\begin{itemize}
  \item important if many variables considered
  \item reduce dimension of feature space: \emph{dimensionality reduction}
  \item fewer correlations to consider, less overfitting
  \item dim reduction: feature \textbf{elimination} or \textbf{extraction} (PCA)
  \item construct new, independent variables from old set by combining them in specific way
  \item drop least important PC - still retain valuable info from ALL original vars
\end{itemize}


%%%%%%%%%%%%%%%%%%%%%%%%%%%%%%%%%%%%%%%%%%%%%%%%%%%%%%%%%%%%%%%%%%%%
%<o><o><o><o><o><o><o><o><o><o><o><o><o><o><o><o><o><o><o><o><o><o>%
%%%%%%%%%%%%%%%%%%%%%%%%%%%%%%%%%%%%%%%%%%%%%%%%%%%%%%%%%%%%%%%%%%%%
\section{Supervised Learning}
\paragraph{classification tasks} predict the probability of x being in class A
\paragraph{regression task} predict mean value of continuous parameter y

\subsection{BDT - Boosted Decision Trees}

\subsection{SVM - Support Vector Machine}
\begin{itemize}
  \item linear hyperplanes
  \item 

\end{itemize}

Sources:
\url{https://www.quantstart.com/articles/support-vector-machines-a-guide-for-beginners/}
\url{https://en.wikibooks.org/wiki/Support_Vector_Machines}



%%%%%%%%%%%%%%%%%%%%%%%%%%%%%%%%%%%%%%%%%%%%%%%%%%%%%%%%%%%%%%%%%%%%
%<o><o><o><o><o><o><o><o><o><o><o><o><o><o><o><o><o><o><o><o><o><o>%
%%%%%%%%%%%%%%%%%%%%%%%%%%%%%%%%%%%%%%%%%%%%%%%%%%%%%%%%%%%%%%%%%%%%
\section{Semisupervised or weakly supervised Learning}

\paragraph{Sources}
- overview: weak supervision
%https://towardsdatascience.com/non-standard-weakly-supervised-classification-problems-51b23211d4bc
%https://www.sciencedirect.com/science/article/abs/pii/S0167865515003505

\subsection{CWoLa - Classification without Labels}
\paragraph{Sources}
%\begin{itemize}
%  \item Presentation: {\verbatim /Users/drdre/Documents/MachineLearning/BOOST_CWoLa_Hunting.pdf}
%\end{itemize}


\paragraph{Use Case:}
\begin{itemize}
  \item extended the bump hunting with ML
  \item classification problem unknown labels or class proportions
  \item or unreliable simulations with statistical mixtures of the classes available
\end{itemize}

\paragraph{Makes use of:}
\begin{itemize}
  \item train directly on data
\end{itemize}

\paragraph{Notes}
\begin{itemize}
\item Need some variable X (e.g. mJJ) in which bg is smooth and signal is localized
\item Need some other variables {Y} (e.g. jet substructure) which may provide discriminating power which may be a-priori unknown.
\item {Y} should not be strongly correlated with X over the X-width of the signal.Or decorrelate (e.g. if we can predict or measure the correlation, that can be subtracted away to create new uncorrelated variables).
\item optimal CWola classifier same classifier as in fully-supervised ML
\item The problem of learning from unknown mixed samples can be shown to be mathematically equivalent to the problem of learning with asymmetric random label noise
\end{itemize}
%%%%%%%%%%%%%%%%%%%%%%%%%%%%%%%%%%%%%%%%%%%%%%%%%%%%%%%%%%%%%%%%%%%%
%<o><o><o><o><o><o><o><o><o><o><o><o><o><o><o><o><o><o><o><o><o><o>%
%%%%%%%%%%%%%%%%%%%%%%%%%%%%%%%%%%%%%%%%%%%%%%%%%%%%%%%%%%%%%%%%%%%%

\section{Unsupervised Learning}
\begin{itemize}
  \item learn structures in data
  \item learn specific data distribution, not know a priori
  \item example methods:
\begin{itemize}
  \item Clustering
  \item density estimation
  \item anomaly detection
  \item PCA (?)
  \item GANs
  \item reinforcement learning
\end{itemize}
\end{itemize}

\section{Reinforcement Learning}
\begin{itemize}
  \item nop data is being used!
  \item an agent learns to perform an action on environment and receives rewards
  \item example algorithms: QLearning, DQN, SARSA
  \item jet grooming through reinforcement learning
  \item qbit control
\end{itemize}


\subsection{Q-Learning}
\begin{itemize}
  \item model-free
  \item Q - quality
  \item goal: develop optimal policy for an agent
  \item exploration (a random compo)
  \item 
  \item trying to learn a policy
  \item Q-function solved iteratively
  \item temporal difference equation - compare reward step t vs t+1
  \item parameters: learning rate, discount, epsilon
\end{itemize}
\subsubsection{Bellman Equations}
\begin{itemize}
  \item depends on state x, action a at timne t
  \item also on reward R
  \item Value is calculated depending opn prior rewards (gamma)
  \item functional equation - solve the Q-function:w
\end{itemize}


\subsection{Variational Autoencoders}
\begin{itemize}
  \item framework using backpropagation-based fct approximators to build
	generative models
  \item weak assuptions, fast training (via backpropagation)
\end{itemize}



\section{Paper Notes - Why does deep and cheap learning work so well?}
1608.08225
Deep Learning in AI is as well understood as learning in the human brain. We
know that certain protocols result in fantastic results in certain tasks, just
like we know that training the human brain with particular methods results in a
specific skill set. But why that is, is not well understood.

At the core of the majority of ML algorithm lies the task of approximating an
arbitrary function, i.e. learning a data distribution following some kind of
structure. This gives three main aspects to shed light on:
\begin{itemize}
  \item[Expressibility] Which class of functions can the algorithm represent?
  \item[Efficiency] Which ressources (ressources, free parameters,...) are
	required to represent the a given function? How does this compare to the
	space of overall available arrangements in that class of functions?
  \item[Learnability] How quickly can my algorithm learn the function
	approximation?
\end{itemize}

Core question in this paper \emph{How is a NN able to learn and approximate
functions efficiently, if the space of possible functions is exponentially
large?}.
Awesome example: Classifiying pictures with Megapixels into cat or dog
categories. Assume each pixel can take 256 values. This gives a possibility of
$256^{1000000}$ pictures! For each picture, we would like to get a probability
of how cat/dog-like the content is. But this means that we have an arbitray
function covering $256^{1000000}$ probabilites!! A much much greater number than
atoms in the universe ($10^{78}$). But NN usually require only millions of
parameters to approximate this well... Hence the term "cheap".\\
In rough terms, the NN break the available exponential space of parameters down
from $v^n$ to $v\times n$ for $v$ being values and $n$ being the number of
inputs. The NN will only work well on a tiny fraction of the exponential input,
but it turns out that laws of physics ensure that this is the fraction we are
interested in, as the input we are interested in is also just the tiny fraction
of all available inputs. The argument is then, that these subset are very
similar, allowing efficient work on these subsets only.\par
Notation: $p(x|y)$ is a prediction problem - given the class $y$, how do $x$
look like, i.e. what is the pdf. $p(y|x)$ is a classification problem - which
class does a given set of $x$ belong to?
It is natural to understand that $p(x|y)$ would come with simplifications. The
class of cat picture exhibits a certain set of characteristics about $x$ which
reduces the relevant space by much. Generally speaking these characteristics
could be symmetries, localities, polynomial functional dependencies etc.
It's harder to understand how $p(y|x)$ with $y=cat$ would simplify, or how we
could apply generic rules to that...
Steps in paper: 
- reformulate $p(y|x)$ with bayes theorem
- Then do a translation of information theory terms into physics terms!
  introduce hamiltonians.
- End up with an Boltzmann-form expression.
- introduce shortened notation
- rewrite Bayes theorem in softmax notation
- investigate which Hamiltonians can be approximated. 
- ML and physics focus on Hamiltonians which are symmetric and sparse and often
  of low-order polynomial complexity
- show that one can approximate multiplication efficiently with small nr of
  neurons
- repeated multiplication and addition then approximates any hamiltonian  in
  taylor expansion
- ML and physics focus on Hamiltonians which are symmetric and sparse and often
  of low-order polynomial complexity
- examples of hamiltonians indeed only range up to small dimensionality! SM
  Hamiltonian has dimensionality of 4, for example.
- locality argument stating that for physical systems only local, neighbouring
  connections have effect
- symmetries like rotational invariance and others reduces number of parameters

\paragraph{What causes deep learning to improve performance?} Or in other words,
what underlying structure of data distributions in the real world cause an
imporvement in NN performance when additional layers are added?
The papers argument involves the hierarchical underlying structure in any
generative process. This follows a Markovian Chain - the state of a system is
determined by its causal predecessor state alone and a transition matrix between
the states. This is iterative.
THe argument then is, that a NN with multiple layers can approximate any
hierarchical structure which can be decomposed. These are ubiquitous in the
physical world, everythiong can be seen from a hierarchical standpoint and
decomposed intop simpler substructures. Each of these substructures can be
approximated by the layers in the NN.

\subsection{Questions}
\begin{itemize}
  \item Where does "Deep" start in Machine learning?
  \item Why is $p(y|x)$ much harder to understand/assign 
  \item Where does divident in Equ. (1) come from? This is an expression for
	$p(x)$. Is that what's called marginalisation?
  \item What exactly does the locality argument state? Couldn't one create a
	counterexample in some image recognition case eg?
  \item What's a Markov Chain?
  \item How is that locality argument anything really? Relation to ML? even in
	physical world: Would it not be possible to have a macro effect from
	non-local influences?
\end{itemize}
\subsection{Insights}
\begin{itemize}
  \item 
\end{itemize}

\paragraph{ML Overview from evernote}

Supervised learning
 - classification
    predict probability of x being in class i 
- regression
predict mean value of continuous label y

Unsupervised learning:
 - learn structure in data:
methods:
clustering
density estimation
anomaly detection
PCA

semi-supervised: 
noisy labels
weakly supervised

generative models
- attempt to learn data distribution
- GANs, VAE
I. Goodfellow taxonomy

Reinforcement learning
no data!
agent to perform action on environment - receive rewards
Q_learning
DQN
SARSA
inverse reinforcement learning


Reinforcement learning: qbit ontol, jet grooming through reinforcement learning

COnsiderations before probl;em to ML?
task mapping to ML paradigm?
model be written analytically>?
dimensionality of data?
data suze?


\section{CNN - Convolutional Neural Network}


\section{Paper Notes - Dynamic Graph CNN for Learning on Point Clouds}
arXiv:1801.07829

Point Clouds (PC) are a geometric representation of data in particular in ciomputer
graphics (or vision). This paper combines successes in CNN with point cloud representation.
Point clouds do not possess topological infoirmation. The authors introduce a NN
module called EdgeConv which allows CNN application on PC.
Paper considers PC classification and segmentation. 
EdgeConv is an NN operation which captures local geometric structures in the PC,
which remain invariant under permutations. Learns edge features, i.e. relations
between points and their neighbors. This approach is based on PointNet.

Related Work:
Handcrafted Features:
extrinsic descriptors use coordinates of PC in 3D space
intrinsic descriptors interpret 3d shape as manifold with a discretized metric
like mesh.
Deep Learning on Geometry: CNN for geometric data
Geometric generative models: adapt VAE, GAN to non-euclidean setting of PC.
Problem is the comparison between input and output, as there is no canonical
order.
Dynamic Graph CNN - DGCNN
- construct a local neighborhood graph
- apply convolution-like operations on graph edges
- An edge is a nonlinear function from a pair of points to real numbers and
learnable parameters.

DGCNN approach for PC
-Edge Convolution
from input points in PC (can also be PC of features in subsequent layer)
-Dynamic Graph Update
Multilayer Perceptron (MLP)

-Properties
-- permutation invariance
because operation based on max() which is symmetric under permutations
-- translation invariance
partially true, as paper choice enables switching off translation dependency
-Comparison to existing methods

Evaluation
- Classification
-- Model Complexity
-- ModelNet40 experiments
- part segmentation
each point in a PC classified into a predefined set of categories
- indoor scene segmentation

Discussion
- new operatopr for learning on PC
- local geometric features important in 3D classification tasks



Questions
Applications of point clouds?
Permutation invariance as quoted on p1:2
What are embeddings?

Figures:
F1 - nice example how algorithm finds "semantic" features in PC
Basically, it decomposes the PC into similar looking objects and marks their
classification in "distance"

F2 - own note: looks similar to RBM? compare this

This being used in HEP: jet reconstruction.

In this paper, we present a new approach for machine
learning on jets. The core of this approach is to treat
jets as particle clouds, i.e., unordered sets of particles.
Based on this particle cloud representation, we introduce
ParticleNet, a network architecture tailored to jet tagging tasks. The performance of the ParticleNet architecture is compared with alternative deep-learning architectures, including the jet image–based ResNeXt-50
model, the particle sequence–based P-CNN model and
the particle set–based PFN model. On both the top
tagging and the quark-gluon tagging benchmarks, ParticleNet achieves state-of-the-art performance and improves significantly over existing methods. Although the
very deep image–based ResNeXt-50 model also shows significant performance improvement over shallower models
like P-CNN and PFN on the top-tagging benchmark, indicating that deeper architectures can generally lead to
better performance, the gain with the ParticleNet architecture is more substantial. Moreover, the high performance is achieved in a very economical way as the number of trainable parameters is a factor of 4 (56) lower in
ParticleNet (ParticleNet-Lite) compared to ResNeXt-50.
Such lightweight models are particularly useful for applications in high-energy physics experiments, especially for
online event processing in which low latency and memory
consumption is critical.
While we only demonstrate the power of the particle
cloud representation in jet tagging tasks, we think that
it is a natural and generic way of representing jets (and
even the whole collision event) and can be applied to a
broad range of particle physics problems. Applications
of the particle cloud approach to, e.g., pileup identification, jet grooming, jet energy calibration, etc., would be
particularly interesting and worth further investigation.


\section{Jet Tagging via Particle Clouds}
arXiv:1902.08570v3

Quark- and gluon-jet separation important for physics at LHC. This paper treats
jets as unordered set of particles, which is different than most approaches
seeing them as ordered structure. This leads to the analogon to point clouds in
3D space used in computer vision - a \emph{particle cloud}. They use the Dynamic
Graph CNN from the other paper (Dynamic graph cnn for learning on
point clouds).  This approach here outperforms state of the art tools!

Jet representation as images comes naturally, given the way they are detected.
Calorimeter cells can be seen as image pixels, their values corresponding to the
deposited energy. This has been well studied and understood using relatively
shallow CNN. Deeper structures might improve things. CNN offer computational
disadvantages and complications when using more information than just
calorimeter.

Another way of representing jets is through their constituent particles. For
deep learning this
requires to order constituents somewhow, commonly according to their transverse
momentum. This is an arbitrary choice and may imply some suboptimal choice.

Jets seen as particle clouds - an unordered, permutation invariant set -  seem
more natural choice. Point clouds are irregular distributed points in space,
with an underlying internal structure, just like jets. 

Convolution is the extraction of features from an input image and can be
understood as the subsequent application of functions. For a pixelised image, it
would map the original pixels onto a smaller dimensional image, preserving
relationships between original pixels 

CNN extremely successful. Key features for their success: 
- convolutional operation exploits translational symmetry of images (shared
kernels across full image). This reduces the number of parameters to be learned
and improves learning, as each weight uses full image. 
- hierarchical approach: convolutions are stacked. subsequent layers learn
different structures of images, from shallow layers learning neighbouring
relations to deep layers learning global properties. 

Here, combine CNN and point cloud. Convolutional operation needs some local
patch to operate - not given in point cloud and needs to be defined.
Furthermore, the convolution form needs to be modified on point clouds to
respect permutation invariance.

That's why they introduce EdgeConv in that aforementioned paper - Edge
Convolution. Each point in the cloud is a vertex in a graph, the connection of
each point with its k nearest neighbours are the edges. This defines a local
patch. With that, the EdgeConv operation is defined. It can be understood as a
perceptron operating on the point and its k neighbours.
These perceptrae can be stacked, making a deep learning structure possible. This
stackability also means that the distances learned by the EdgeConv operation
change with each layer. The parameters of the Network are thereby updated. This
results in a Dynamic Graph CNN.

ParticleNet is an adaption of the EdgeConv and DGCNN structures. 
One Building block is the EdgeConv operation. This operation goes as follows:
- find k neartes neighbors (k is a hyperparameter)
- construct edge features from the "features" (?)
- feed to 3 layer multilayer perceptron - linear layer, batch norm,lisation and
ReLU. 
- 

Approach outperforms staet of the art technology!


Q: 
- How is the cloud permutation invariant? 
- Fig 1: features as input to edge features?
- adaptive learning rate observed before? (p4)
-- warm up
-- cyclical
-- depending on optimiser
-- ADAM not necessary, since it is adaptive. check it!
- cross entropy loss is constructed how? what's the truth info?
- input to algorithm given by kinematic variables?

