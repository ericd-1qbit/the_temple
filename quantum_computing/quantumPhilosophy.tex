%!TEX root =../main/main.tex 
%Author: Eric Drechsler
\chapter[Quantum Philosophy]{Quantum Philosophy}
%+++++++++++++++++++++++++++++++++++
\label{ch:quaphil}

\paragraph{Random Notes}
\begin{itemize}
  \item machine conciousness - see Darwiche 2018, Covey 1989
  \item Monism, Physicalism (Emergence from complex systems) versus Dualism
(physical world through thought) \end{itemize}


\section{The Big Questions}
\subsection{Quantum mechanics and Zombies}
A frequent argument in discussions about consciousness seems to be: If there is
nothing, then what's the difference between a Zombie version - i.e. a particle
by particle reconstruction of your full self - and you?\par
I would argue that quantum mechanics might play a key role here. It is not
possible to build an exact copy of anything down at the quantum level. The
no-cloning theorem forbids exact copies of quantum states. On a macrocopic level
a copy might be created, but the exact setup of entanglement and superposition
of states differs. There is a fundamental restriction of the level of
cloneliness. An exact copy of oneself with all particle arrangements according
to the blueprint could be made with the exception of quantum level effects.
This would entail the question: does consciousness form in these.

\paragraph{Questions}
\begin{itemize}
  \item check exact definition of no cloning theorem
  \item can two quantum states ever be the same?
\end{itemize}
