%!TEX root =../main/main.tex 
%Author: Eric Drechsler
\chapter[Quantum Computing]{Quantum Computing}
%+++++++++++++++++++++++++++++++++++
\label{ch:quacom}

\section{Quantum Machine Learning}
\subsection{Roadmapping Workshop}
%!TEX root =../main/main.tex 
This section is a summary of the reading done for the Quantum BC Roadmapping
workshop. Purpose of this workshop is to lay out a plan for the future
development of Quantum Computing in BC. Together with the participants, a
discussion will be held to define goals and clarify the current state. The
following subsections are to facilitate the discussion.

\subsection{QML - Definition}
Defined as the integration of quantum algorithms in the sector of machine
learning. Machine learning tasks are solved sing quantum algorithms. There are
overall 4 different approaches to combine these domains, as depicted in
\ref{fig:algvsdata}. QML is currently limited by the amount of data which can be
processed. As opposed to the classical counterpart, qbits and q gates are
utilised to:
\begin{itemize}
    \item speed up certain parts of the algorithm's computations
    \item improve data storage access
    \item enhance model expressability
\end{itemize}
QML algorithms include hybrid approaches, which utilise classical processing and
outsource computationally expensive parts to QPUs. 

Careful, there are other definitions and aspects which are part of QML, which
are \textbf{not} part of this discussion:
\begin{itemize}
    \item classical ML of quantum data
    \item quantum learning theory, the connection between learning theory and
    quantum information
    \item research on similarities between learning and physical systems
\end{itemize}

\begin{figure}[ht!]
     \centering
    \includegraphics[scale=0.9]{figz/QML_approaches}
     \caption{}
  \label{fig:algvsdata} \end{figure}
\clearpage

\subsection{Basic Concepts}
Core procedure:
\begin{itemize}
    \item Encode data into quantum information processable form, such that it
    can be used on a QPU. 
    \item process information  by running algorithm
    \item result is given by measurement of final qbit states
\end{itemize}

\subsection{Advantages}
- quantum feature maps
Furthermore, to eventually obtain any quantum advantage, 
one should search from the set of classically intractable feature maps

\subsection{Supervised Learning}
\subsubsection{Classification}
The most prominent methods are quantum kernel estimates and variational quantum
classifiers.

\subsection{Unsupervised Learning}
\dots

\subsection{Taxonomy}

\paragraph{Quantum State Amplitudes} can be used to compactly encode exponential
number of information. A state with $n$ qbits has $2^n$ complex amplitudes,
allowing to design algorithms, which grow polynomially in $n$, which corresponds
to logarithmic growth in the number of amplitudes. Many of these are based on
the \textbf{HHL-algorithm}, a q-alg for solving linear systems of equations. In
a nutshell, this allows for matrix inversion which grows logarithmically as
opposed to quadratically, the most efficient known classical method. The
inversion of matrices can be used for any machine learning task relying on
solving a system of linear equations, like the least-squares versions of
\textbf{SVMs, linear regressors, Gaussian processes}. Challenges: State
preparation.

\paragraph{Grover Search} is a quantum search algorithm, which provides a
quadratic speed up of unstructured search problems. It's generalisation is
amplitude amplification. These can be used to speed up ML algorithms which are
based on such unstructured searches, eg. k-medians and k-nearest neighbour
algorithms, quantum perceptron.

\paragraph{Reinforcement Learning} depends on an agent which interacts with the
environment and receives rewards depending on its actions. In the q-realm, this
agent either possess quantum processing capabilities or probes the (classical)
environment through superposition, which both can lead to quantum speed ups. 

\paragraph{Quantum Annealing}  is a technique to determine minima of an
optimisation problem.   A problem is encoded into a Hamiltonian of a system.
This system is evolved over time; it follows the Schroedinger equation in its
time evolution. Initially, the system starts as equal superposition of all
states. Over time, it evolves into the ground state. The crux is to formulate
the problem at hand such that the ground state reflects the solution to a (ML)
problem. In particular, the form of problem must be a QUBO. 

\paragraph{Quantum Sampling} can speed up computationally hard sampling
procedures, which are at the core of many ML applications - Boltzmann Machines
as a particularly well-known example. The task is to estimate averages over
probabilistic distributions, which can be a hard task for generic probabilistic
models and is usually solved with Markov chain Monte Carlo (MCMC). Quantum
Annealing offers a different approach, here samples are drawn reluying on an
actual pysical process like quantum annealing, which naturally provides samples
following the Boltzmann distribution. 
Aside from Annealing a universal QC can also be used for probabilistic sampling:
a "thermal state" is prepared on the QPU and then measured, resulting in a BM
sample. This state is prepared using quantum enhanced Markov logic networks.

\paragraph{Quantum Neural Network} describes a class of algorithms combining
aritifical neural networks with quantum information advantages to gain speed up
and higher efficiency. Most of these are implemented as feed-forward
architectures.
Examples of QNN models are: 
\begin{itemize}
    \item Quantum perceptrons: the qbits take the place of neurons. The main
    problem here is the non-linearity required for neural activation. Quantum
    theory is linear. A practical implementation has been proposed using circuit
    based on variational quantum circuits and quantum phase estimation.
    \item Quantum Networks: theoretically generalisation of the classical neuron
    leads to general form of unitary gates. This can then be applied to a
    variety of different qbits. The learning is then still based on VQC. 
    \item Quantum Networks for algorithm design: keep the interactions between
    qbits as learnable parameter - use ML to learn the best algorithm!
    \item Quantum Associative Memory: a circuit based quantum ciomputer,
    simulating associative memory - based on Grover-like quantum search
    algorithm. The search retrieves the strate closest to the given input. Not a
    neural network in the Hopfield sense.
\end{itemize}

\paragraph{Hidden Quantum Markov Models} are the quantum equivalent to Hidden
Markov Models, a technique to learn sequential data. Density matrices are used
to express probabilities. They are tied to Bayesian Inference, which is an
interesting connection.

\paragraph{Quantum Quantum Learning} refers to the system and the computation
device being quantum states. Examples are "learning" a quantum state without
finding a classical description first. Clustering is another example allowing a
complete quantum implementation.

\subsection{Challenges}
\subsubsection{HHL-based algorithms}
\textbf{State preparation} is a crucial bottleneck for any algorithm based on
quantum state amplitude manipulation. This relates to \textbf{qRAM}.

\subsubsection{Quantum Sampling}
\textbf{State preparation} for non-annealing based sampling, "thermal state
preparation protocol". 

\subsection{Local Projects}
 

\section{General}

\subsection{Definitions}
\subsubsection{NISQ} 
Noisy Intermediate Scale Quantum machines are computers with $\mathcal{O}(100)$
qbits, which are not error-corrected (noisy). These perform imperfect
operations and have a limited decoherence time.

\subsubsection{Fault Tolerant } 
These machine have $\mathcal{O}(10^{6})$ (millions) qbits, which are
error-corrected and have a long decoherence time compared with the required
operations. This allows for circuits with great depth. These will be able to
solve the hard problems QC is famous for - SHor, GRover, etc.

\subsection{Local Projects}
\paragraph{TRIUMF}
\begin{itemize}
  \item using variational autoencoders: classification and synthetic data generation (neutrino detector outputs)
  \item efficient qubit encoding for calculating molecular energies with variational quantum eigensolver
  \item graph-theoretic methods to reduce problem size of QUBOs for QAnnealing
  \item Extrapolating Nuclear Many-Body Calculations with Constrained Gaussian Processes
  \item Water-Cherenkov Machine Learning
\end{itemize}


\paragraph{UBC}
\begin{itemize}
  \item 
  \item 
\end{itemize}

\paragraph{SFU}
\begin{itemize}
  \item classification
  \item particle tracking
\end{itemize}


\subsection{Companies to check out}
\begin{itemize}
  \item Strangeworks
  \item particle tracking
\end{itemize}


%Research at Helmholtz
% Improving the variational quantum eigensolver (Bayesian
%optimization with Gaussian processes, gradient optimization)
% Applying variational quantum eigensolver to lattice gauge
%theory simulation
%Solving real-world problems with quantum annealing and QAOA
%(quantum approximate optimization algorithm)
%Other news
% Quantum computing retreat on 28 November with faculty from
%across the country – what should TRIUMF’s role be for quantum
%computing in Canada?
% On-going lecture series about quantum computing hardware
%and how to build qubits
%TRIUMF:
%Helmholtz:
% Constantia Alexandrou named Helmholtz International Fellow
%and will be visiting DESY; will contribute to development of
%quantum computing at Helmholtz
% Jülich is creating a platform with different quantum computers
%(from Kerstin: more news next time)

\subsection{Unsorted Notes}

\paragraph{Skimming notebooks}
\begin{itemize}
  \item entanglement: state is not representable by tensor-product of single, pure states and always depends on basis (partition)
  \item Bell-state prime example: $\sqrt{2}^{-1}(\ket{00}+\ket{11})$ - cannot be created with linear combination of basis states
  \item ancilla qbit: extra, unused (?) qbit
  \item connectivity and nr of nodes in QUBO do not necessarily influence gap
	size of min E (adiabatic)
  \item qbit error correction principle: make multiple "copies" of a
	measurement. instead of measuring 0 and 1 measure 000 and 111. observe spin
	flip. more difficult due to no-cloning theorem.
  \item code distance: how many errors can your code detect?
\end{itemize}

\paragraph{Tracking}
\begin{itemize}
  \item tracking: matching hits to trajectories, fitting trajectories to understand particle properties
  \item general difficulties: secondary vertices, multiple scattering, holes, double hits
  \item current algorithms complexity: $\mathcal{O}(N^2)$ (N: number of hits)
\end{itemize}
\begin{itemize}
  \item fast track finding based on NN
  \item form doublets of hits (with selection criteria)
  \item do a binary classification: which doublets are ok to keep? Define Energy function (Stimpfl-Abele)
  \item doublet definition hard - high pt and small densities necessary
  \item in hepqpr.qallse - use triplets and combine to quadruplets, quintuplets
\end{itemize}

\paragraph{TrackML specifics}
\begin{itemize}
  \item $1\%$ missing hits
  \item $10\%$ double hits
  \item $15\%$ noise hits
  \item $\mu>140$, number particles $11.000$
  \item $1-12$ hits per particle, average 12
  \item therefore $110.000$ hits per event
\end{itemize}
\paragraph{Doublet creation}
\begin{itemize}
  \item $\Delta\phi<0.15$ between hits
  \item $\Delta\theta<0.10$ between hits
  \item less than 4 detector layers between hits
  \item ordered towards outside
\end{itemize}

\paragraph{Other limitations}
\begin{itemize}
  \item track origin always $(0,0,0)$
  \item if track finding with with only doublets - always a doublet
  \item tracks need to be high pt
  \item more false positives than usual methods
  \item works up to 200 tracks - less than 4k hits
  \item only central barrel used from track detector
  \item 
\end{itemize}

\paragraph{DWave}
\begin{itemize}
  \item QUBO/Ising translated to BQM internally
  \item Josephson Junctions - two superconductors with insulator in the middle
  \item superconducting qbits at DWave - cooper pairs spread in material
  \item cooper pairs oscillating between sides
  \item different qbits according to shape of potential bias - transmon, gmon
  \item charged qbits best for universal QC
  \item flux qbit for annealing - double well potential by JJ+inductivity
  \item short coherence times? no killer for annealing, just decreasing efficiency
  \item next evolution of machines - Chimera 16 to Pegasus 6 to Pegasus 16
  \item Pegasus 16: 5.5k qbits, 40k junctions, 15 connections per qbit
\end{itemize}

From Evernote:
- toolkit: ocean tools, open source
- problems on dwave can be understood via graph theory
- QUBO:
-- binary variables
-- objective function to minimise, its minimum leading to desired result
-- constraints are added as penalty to OO
--- Lagrange parameters use to tune iompact of constraint
- sampling: from all solution, pick one with lowest energy values
-- all solutions characterise the energy landscape


\subsection{DWave Advantage}
Newest Quantum Computer introduces new QPU architecture: Pegasus. Advances in
QPU fabrication technology allows for a higher connectivity between qbits. On a
QPU not all qbits are connected to each other, instead the connectivity follows
a certain topology. How a problem is embedded on the QPU is crucial - not only
the qbit count matters but also the topology. In a sparse QPU structure (one
where not all qbits are connected) certain QUBO terms can only be embedded by
using logical qbits: a qbit formed by coupling two qbits strongly, making them
act as one. This allows certain graph edges and nodes to be interconnected on
the chip. A variable represented on the chip is therefore not a single qbit but
a logical qbit (set of strongly connected qbits). The procedure of doing this
kind of problem embedding onto logical qbits to fit the architecture is called
minor embedding.
The size of a problem a QPU can process depends on its qbit count but also on
the connectivity, or the working graph - the available qbits and couplers
between them.

Previous DWave generations have 2000+ qbits, the Advantage system 5000+. The
connectivity between those is just as important. The topology is denoted as
either CN or PN, where C and P stand for Chimera and Pegasus respectively, and N
is the number of unit cells in a grid. C16 is a lattice of 16x16 unit cells. C16
consists of 2048 qbits (16x16x8). A unit cell is NOT one qbit. A unit cell is a
block of more densely connected qbits. The blocks are connected in the CN
lattices.
Hardware limitations - manufacturing, cryogenic - the count of avaiolable qbits
is lower than the lattice suggests. 

% Continue at: ## Test Case: Embedding Random Graphs 

\paragraph{DWave Hybrid Solver}
\begin{itemize}
  \item new (2020) framework to enable parallel workflows invoking various
	samplers
  \item enables decomposition of large QUBOs (cut at 10k?) into set of smaller
	samples
  \item multiple solution techniques can run in parallel in branches
  \item for example: have global tabu search (classic qbsolv) - can be
	interrupted by another, parallel running branch
  \item branches (parallel solution runners) can be divided into Decomposer
	(split current sample according to some criterion),
	Sampler (QPU, SA, ...) and Composer (user defined criterion to select from
  current samples)
\end{itemize}

\paragraph{Quantum annealer}
\begin{itemize}
  \item adiabatic theorem: system remains in ground state if 
  \begin{itemize}
	\item perturbations slow
	\item large enough energy gap to next excited state
  \end{itemize}
  \item  $H(s)=A(s)H_0+B(s)H_p$ - initial Hamiltonian $H_0$, problem Hamiltonian $H_p$
  \item anneal time $s=t/t_f$. amplitude A lowered, B raised - introduce problem, remain in ground state
  \item problem formulation on annealer: quantum machine instructions as quadratic unconstrained binary optimisation QUBO
  \item objective function:
\end{itemize}
\begin{align}
  O(a,b;q)=\sum_{i=1}^{N}{a_iq_i}+\sum_{i=1}^{N}\sum_{j=1}^{N}{b_ijq_iq_j}
\end{align}
\begin{itemize}
  \item minimum gap: (E vs time spectrum) point where ground state energy approaches next energy level closest
  \item complex problems:  many energy states close to minimum, several minima...
  \item thermal fluctuations can lead to jump out of minimum energy
\end{itemize}
\begin{itemize}
  \item 
\end{itemize}

\paragraph{QC general}
\begin{itemize}
  \item dimensionality argument: classical - cartesian product, quantum - tensor product
\end{itemize}
\begin{align}	
  \mathrm{classical:}\quad& \dim{V \prod W}=\dim(V)+\dim(W)
  \mathrm{quantum:}\quad&   \dim{V \otimes W}=\dim(V)\dim(W)
\end{align}
\begin{itemize}
  \item $n$-qbit system: $2^n$ amplitudes
\end{itemize}

v

\paragraph{Quantum Neural Networks}
\begin{itemize}
  \item see reading list, two related items: 1912.12486, 1811.02266
  \item similar to qRAM
  \item state preparation via hypergraph
  \item defered measurement - two ways
\end{itemize}

\paragraph{Quantum Circuit Learning}
\begin{itemize}
  \item hybrid algorithm - quantum+classical component
  \item low depth quantum circuit (better for NISQ)
  \item ideal - Quantum-Phase-Estimation: Harrow-Hassidim-Loyd Algorithm
  \item low depth circuits:
\begin{itemize}
  \item  QVE/VQE (Variational eigensolver)
  \item  QAOA (approximate optimisation algorithms
\end{itemize}
  \item QCL basic idea: problem divided into 2 parts
  \item encode problem in A (hermitian matrix) and optimise expectation value
  \item optimisation wrt parameter dependant state $\ket{\psi(\theta)}$
  \item algorithm
\begin{itemize}
  \item[1] prepare state $\ket{\psi(\vec{x})}$ with unitary operations from initial encoding data points $x_i$
  \item[2] parameterise this state with unitary operations and tunable parameters $\vec{\theta}$
  \item[3] measure expectation value of observable
  \item[4] calculate loss function from predicted labels (0,1) on $\vec{x}$
  \item[5] embed 2-4 in classical loop to minimise loss function by varying $\vec{\theta}$
  \item[6] evaluate on test/validation set
\end{itemize}
  \item quantum advantages: non-linearity can be modelled without high cost/number of nl functions (classical)
  \item kernel-trick-method
  \item avoids overfitting
\end{itemize}

\paragraph{Variational Quantum Classification}
\begin{itemize}
  \item QCL paper uses Qulacs framework
  \item also available: variational quantum classification (VQC) part of QiSkit
  \item see presentation
\end{itemize}

\paragraph{Qiskit notes}
\begin{itemize}
  \item shots - number of experiments run on QPU for one measurement - how often same circuiot repeated
  \item also available: variational quantum classification (VQC) part of QiSkit
  \item see presentation
  \item variational form RyRz - number of parameters $n\cdot(d+1)\cdot2$ (number qbits, depth)
  \item number of qbits = number of featrures
  \item feature maps - how to put classical data on QPU
\end{itemize}

\paragraph{Easy ways to explain}
\begin{itemize}
  \item qbits - lightbulb in room
\end{itemize}a


%https://interestingengineering.com/the-race-for-quantum-supremacy-7-reasons-why-people-are-so-excited-about-quantum-computers

- quantum supremacy: a programmable quantum device solves a problem no classical
  computer can solve in any feasible amount of time
-- for google: 200 seconds  vs 10k years
- computational power based on QM principles of superposition and entanglement
- superposition: a system is not fixed to a specific state but can "be" in
  several states at once, governed by a prob function
-- QC perform operations on the object's state probability
- example: Maze.
-- classical, every possible path explored until solution. One by one.
-- QC: all paths at once
- domain of AI/ML: computational power might enable step beyond current "narrow"
  AI towards general AI
-- weather forecast - huge impact on society
-- drug development: simulate molecule interactions much more efficiently
- cryptography: old protocols might be broken by Shor Factoring alg. new
  protovcols based on quantum much safer.


\paragraph{Quantum Generative Models and VAE}
%DWAVE talk 2018_qVAE_DWace_QC_for_HEP
\begin{itemize}
  \item hybrid approach - classical/quantum:
  \begin{itemize}
	\item NISQ devices to process encoded/compressed representations of data
	\item learn these encodings through classical Deep Learning
  \end{itemize}
  \item[$\rightarrow$] Quantum Generative Models with latent variables
  \item goal: reproduced a data distribution by approximating it through a model
	which marginalised the original distribution in latent variables (hidden or
	unobserved variables in original distribution)
\item sampling with QA to replace CPU-inbtensive MC simulation
\item priors estimated with quantum computers might be perfect for q processes
  like in HEP
\item qGenerative models: computational advantage?
\item what has been shown so far:
  \begin{itemize}
	\item qVAE are classes of state-of-the-art generative models - performance
	  en par for complex datasets;
	\item qVAE can be trained with QA on non-trivial datasets - MNIST - TODO:
	  HEP? 
	\item Path to quantum advantage in realistic applications (with NISQ)
  \end{itemize}
	\item Examples of Generative Models trained on QA:
  \begin{itemize}
	\item Quantum Boltzmann Machine: non-hybrid. connectivity of qbits major
	  problem
	\item Quantum Assisted Helmholtz Machine: hybrid. lacks well defined loss
	  function and is inefficient in training
	\item QVAE: hybrid, efficient training, state of the art generative
	  modelling
  \end{itemize}
\item qVAE: 
  \begin{itemize}
\item[1] classical AutoEncoding structure - fwd-propagation through DNN
\item[2] quantum prior: generative process - sample from quantum Boltzmann
  distribution
\item[3] hybrid framework: efficient training with stochastic gradient descent
  \end{itemize}
  \item quantum in qVAE: the prior is implemented as qBoltzmann Machine
  \item problem: must generalise reparametrisation trick to discrete variables
  \item developed at DWave: DVAE by back-propagation through smoothing distributions or
	through smoothed, continous variables
\item q Boltzmann Machine:
  \begin{itemize}
	\item uses a thermal distribution of a quantum spin system to reproduce data
	  distributions
  \end{itemize}
\item path towards quantum advantage:
  \begin{itemize}
	\item improvement of QPUs for better sampling quality
	\item increase generative capacities by exploiting representative quantum
	  distributions and larger number of latent variables
	\item lots of classical ressources
  \end{itemize}
  \item contrastive divergence
  \item training a BM: clamped and unclamped values
\end{itemize}

\section{qRAM}
An integral part of quantum Machine Learning tasks is to translate classical
data to the quantum circuit. This state preparation is also called quantum RAM.
The term encompasses all circuitery and hardware needed to prepare data for
further processing on the QPU. This is one crucial component of any qAlg, as the
efficiency of loading data is a factor in an speedup estimation.
Reminder: a qbiot is given as a superposition of basis states with an amplitude
per state. For $n$ qbits we have $2^n$ amplitudes.
There are two ways to encode data into a quantum state:
\begin{itemize}
  \item[\textbf{Amplitude encoding}] Take classical data and encode them into
	amplitudes, i.e. by rotations on the Bloch Sphere.
  \item[\textbf{Basis encoding}] Each data point is transformed into binary
	strings, which are the qbit configurations. A superposition of those is a
	quantum state.
\end{itemize}
Now what is an actual quantum Random Access Memory? It's defined as a encoding
procedure which loads the given classical data into a superposition quantum
state.
\paragraph{Classical RAM} A classical RAM can be visualised as a grid of cells.
A RAM query is a call to a specific address in that grid for its content.
\paragraph{qRAM} A qRAM can retrieve multiple locations at once. This is done by
performing a query in superposition! The query itself is a quantum
superposition.
There are two types of qRAM:
\begin{itemize}
  \item[explicit] Data is directly encoded into physical qbits. A quantum
	circuit is used to extract that data. Easy to change memory contents. But need a
	whole lot of qbits just to hold data.
  \item[implicit] Design a circuit such to extract the information directly.
	Make a query of all states at once... TODO: didnt understand well.
\end{itemize}
A qRAM suffers under error rates and needs active error correction. Note that QQ
QML algorithms don't really need qRAM. Also there's papers of CQ algorithms
which avoid the use of qRAMs.



\section{QML in HEP paper}
%Quantum Machine Learning in High Energy Physics - arXiv:2005.08582v1
- a review paper - great for talks!
- how can QC learn from data?
-- learning theory adapted to quantum realm - new paradigm
-- find quantum algorithms which speed up ML (computational complexity)
-- ML applications on NISQ devices
- circuit-based QC: predict a QML model, trained classically
-- see Circuit-centric quantum classifiers
-- Fast quantum learning with statistical guarantees.
- qu annealers: optimise a classical model 

\paragraph{Quantum circuits as trainable models}
- ML model can be understood as function depending on input x and tunable
  parameters $\theta$
- one measurement of a quantum circuit is a probabilistic procedure. many
  measurements (shots) are deterministic and determine the expectation value
- quantum model defined by shots of quantum circuit
- parameters and input are decoded into the quCircuit by rotations of qbits
  (physical operation)
-- profound conceptual differences between free para and input - see paper 39
- quCircuit is combination of all gates (encoded data, parameters) - measurement
  expectation value depends on the model parameters
- quCircuit wrapped in classical loop to optimise parameters such that model
  describes data best
-- measured with loss function
-- called variational or parameterized or QNN.
-- use gradient free or finite-difference based opt. methods
- important recent result: gradient can be calculated by using macroscopic
  "finite differences" (compare to Xanadu talk method). 
-- can thereby compute gradients in variational circuits
-- differentiable programming
- probabilistic nature of quantum circuits - use as ansatz for generative
  adversial networks!

\paragraph{Quantum annealers as optimisers}
- quantum part not used for prediction but for training (tuning parameters)
-- problem is to rephrase problem (i.e. loss function) into QUBO format
- first application: simple perceptron ensembles can be rephrased as QUBO!
-- QBoost algorithm! check it out.
- QA as samplers for RBM - qRBM. :)

- examples: diphoton classification and RBM training

\paragraph{RBM}
- generative model approximating a target distribution
- set of visible variables and set of hidden variables
- trained with gradient descent by updating weights and biases


\paragraph{Quantum Circuit Applications}
- VQC, QCL


Q:
- Why does the maximum speed of evolution decrease with system size? (p4)

- what is learning theory in quantum setting (p4)
see 29 \& 30

- Discuss gradient-free/finite difference methods. why not gradient? understand
  again Xanadu example (p6)

- what's differentiable programming? (p6)

- clamped and unclamped values? (p10)

- why input, edge and node NETWORK, not just layer? (p12)

- what's a Tree Tensor Network (p12)

- any info on ADAM optimiser (p13)?




%Notes from Quantum Application Workshop
Quantum Applications at DWave

Which problems can be solved?
Optimisation - Traffic rutng, Graph partitioning
Probabilistic ML - Clustering, RBM
Constraint Satisfaction Problem - Scheuliungm Resource allocation

Early APplication
- Materials properties, opt., ML, cyber security


A case for QA for regression
- Roman Krems
- focus on small data
- probabailistic learning base of bayesian inference


1Qbit
- Phase estimation algorithms (PEA) - less measurements
- VQE (less qubits and shallower circuits)

\paragraph{Quantum Autoencoders to denoise quantum data}
%arXiv:1910.09169v1
A key to computational advantage with QC is the creation and manipulation of 
entangled states. Experimental noise is a limiting factor. Autoencoders are known 
classical NN architectures which allow to denoise classical data in an unsupervised manner.
This paper constructs a quantum AE from quantum NNs to denoise quantum data.\\
Denoising protocols for quantum states are hard to come up with, they require
knowledge of all noise sources and correction methods which do not change the 
characteristics of the original state. Also, often no denoised references are available.
Chosen QNN is open source, computational complexity scales linearly with depth per epoch, 
cost function has clear operational meaning.  
The layer in the AE where inputs are mapped to the latent space is called bottleneck in this paper.

%Stuff to discuss
- BirdNET app for birdsong identification
- discuss expression 1
- What's the meaning of the fidelity (wiedergabetreue)?  
- p2, probability for spin flip - 


\subsection{An efficient qu algorithm for generative machine learning}

Questions:
What is the reasonable assumption on complexity theory?
How does the training work?
What's a generic form of correlator?

Paper summary: Standard generative models can be expressed as factor graphs,
with vertices variables and functions. The quantum version of that is the Tensor
Network state - with hidden and visible variables. 
Translate this TNS to their model - the QGM.

The Zist: they prepare a network state Q(Z) by applying invertible,
parameterised operations on qbits and then measure a subset of them. this can
model a probability distribution. This can be trained efficiently.

They show:
- any factor graph is a special case of QGM. 
- QGM is exponentially more powerful than factor graphs for representing pdfs
- there are problems for which the QGM scales polynomial while classical
algorithms don't

\section{Quantum Boltzmann Machine}
%arXiv:1601.02036v1
\newcommand{\be}{\begin{equation}}
\newcommand{\ee}{\end{equation}}
\newcommand{\ba}{\begin{eqnarray}}
\newcommand{\ea}{\end{eqnarray}}
\renewcommand{\bf}{\mathbf}

\newcommand{\nn}{\nonumber \\}
\def\p{\partial}
\def\tr{\text{Tr}}
Introduction

Machine learning aims to learn data distribution by adjusting parameters in a
model. If model distribution is sufficiently similar to data distribution, one
can sample from it to generate examples following the data distribution.

Quantum Machine learning algorithm are only used to facilitate training, i.e.
the model is classical, the training is sped up. The QBM
actually incorporates quantum effects in the model - a \emph{quantum
probabilistic model}. It does this by using the Boltzmann Distribution of a Hamiltonian describing a quantum
system.
\subsection{Boltzmann Machine}

A classical Boltzmann Machine consists of visible and hidden nodes $z_a$. It's
Energy is given by 
\ba
E_{\bf z} = - \sum_{a} b_a z_a - \sum_{a,b} w_{ab} z_a z_b. \label{Ising}
\ea

This is an Ising model and it is defined by the quadratic energy function over binary units. 
The marginal distribution
  \be\label{PV}
  P_{\bf v} = Z^{-1} \sum_{\bf h} e^{- E_{\bf z}},\ \qquad Z = \sum_{\bf z} e^{-E_{\bf z}},
  \ee

Training goal: determine the parameters of the BM (bias, weights) such that
model an data distribution as close as possible. This is a minimisation of the
avergae negative log-likelihood.
\ba 
{\cal L} = -\sum_{\bf v} P_{\bf v}^{\rm data} \log P_{\bf v}\\
{\cal L} = -\sum_{\bf v} P_{\bf v}^{\rm data} \log {\sum_{\bf h} e^{-E_{\bf z}} \over \sum_{\bf z'} e^{-E_{\bf z'}}}
\ea

\ba
\p_\theta {\cal L}
&=&   \sum_{\bf v} P_{\bf v}^{\rm data} {\sum_{\bf h} \p_\theta E_{\bf z} e^{-E_{\bf z}} \over {\sum_{\bf h} e^{-E_{\bf z}} }} - {\sum_{\bf z} \p_\theta E_{\bf z} e^{-E_{\bf z}} \over \sum_{\bf z} e^{-E_{\bf z}}} \nn
&=&  \left[ \sum_{\bf v} P_{\bf v}^{\rm data} \langle \p_\theta E_{\bf z}\rangle_{\bf v} - \langle \p_\theta E_{\bf z} \rangle \right], \nn
&=&  \overline{\langle \p_\theta E_{\bf z}\rangle_{\bf v}}
- \langle \p_\theta E_{\bf z} \rangle,
\ea

\textbf{Clamping} refers to fixing the visible variables to data in the averaging.
\ba
\delta b_a &=&  \eta \left( \overline{\langle z_a \rangle_{\bf v}} - \langle z_a \rangle \right),\\
\delta w_{ab} &=&  \eta \left( \overline{\langle z_az_b \rangle_{\bf v}} - \langle z_az_b \rangle \right).
\ea

The parameter gradients are expressed in terms of differences between clamped
(visible vars fixed to data) and unclamped, free averages - unclamped and
clamped terms are also called positive and negative phase. Estimating these
averages is done through sampling! Efficient way of sampling = efficient way of
training.

\subsection{Quantum Boltzmann Machine}
Now the Energy function in \ref{me} is replaced by a Hamiltonian describing a
N-qubit system with two states, i.e. dimensionality $2^N$.
THe diagonal entries of this Hamiltonian are exactly the energy values of all
possible $2^N$ states of the binary units. 
For that, the $z_a$ are replaced by $2^N\times2^N$ matrices
\be
\sigma^z_a \equiv \overbrace{I\otimes ... \otimes I}^{a-1}\otimes \sigma_z \otimes \overbrace{I \otimes... \otimes I}^{N-a} \label{sigma_def}
\ee

$I$ is the identity matrix, $\sigma_Z$ is the Pauli-Z Gate introducing a phase
flip. 
This is a fully connected BM where each visible and hidden node is connected
with each other and amongst themselves. 
Eigenstates of this are denoted as $\ket{\bf v,h}$
Then define density matrix, marginal B probability, prpojection operator and add
transverse field. This transverse field adds transverse spin components. 
The new H now has eigenstates which are superpositions of the initial $\ket{\bf v,h}$.
Wirth this new H, the parameters are trained again to match visible distribution
to data. A complication arises as the Operators in the do not commute anymore
and therefore can't be obtained like seen in the classical case.
One of the resulting terms can be reexpressed and sampled. The other term not.
Instead of minimising the NLL, the minimisation target is an upper bound on the
NLL. 

\subsection{Introducing upper bound}
Using the Golden-Johnson inequality, the trace in the NLL derivatives can be
reexpressed. The new form introduces a clamped Hamiltonian $\mathbf{H_v}$. 
With this upper bound a new optimisation target is defined. The paper then shows
examples on how this compares. The transverse field cannot be learned with this
upper bound.

\subsection{(Semi)Restricted QBM}
The positive phase calculation requires clamping of visible variables to data
and thenn sampling. This can become expensive. If however, the hidden variables
are not connected to each other, the calculations can be carried out exactly,
i.e. expectation values can be computed exactly.

\subsection{QBM on a Quantum Annealer}
Time dependent Hamiltonian implemented on QA:
\be
{\cal H}(s) = - A(s)\sum_{a}\sigma^x_a + B(s)[\sum_{a}h_i\sigma^z_a {+}
\sum_{a,b} J_{ab}\sigma^z_a\sigma^z_b ], \label{HP}
\ee
In general: a quantum annealer with a linear annealing schedule does not return
a BM distribution. But with certain modification to the annealing schedule, a BM
distr. can be achieved. The dynamical slow-down and freeze-out need to happen in
a short time.

Notation:
$z_a=(z_v,z_i)$
with v for visible, i for hidden.
$z_a\in\{-1,+1\}$
with $a=1,...,N$
Vector
notation $\mathbf{v,h,z}$ denoting a visible, hidden, BM state respectively.

Hamiltonian Parameters $\theta\in{b_a,w_{a,b}}$

Questions/Key Points:
- How is Quantum advantage encoded in the model?
- In what sense is Boltzmann distribution used on top of a quantum Hamiltonian?!
- What's a Boltzmann average?
- The translation of the z to sigma z - why?
- Why is the transverse field added?
- How does sampling step actually work?

TODO: 
- Calculate Equation 6