%!TEX root =../main/main.tex 
This section is a summary of the reading done for the Quantum BC Roadmapping
workshop. Purpose of this workshop is to lay out a plan for the future
development of Quantum Computing in BC. Together with the participants, a
discussion will be held to define goals and clarify the current state. The
following subsections are to facilitate the discussion.

\subsection{QML - Definition}
Defined as the integration of quantum algorithms in the sector of machine
learning. Machine learning tasks are solved sing quantum algorithms. There are
overall 4 different approaches to combine these domains, as depicted in
\ref{fig:algvsdata}. QML is currently limited by the amount of data which can be
processed. As opposed to the classical counterpart, qbits and q gates are
utilised to:
\begin{itemize}
    \item speed up certain parts of the algorithm's computations
    \item improve data storage access
    \item enhance model expressability
\end{itemize}
QML algorithms include hybrid approaches, which utilise classical processing and
outsource computationally expensive parts to QPUs. 

Careful, there are other definitions and aspects which are part of QML, which
are \textbf{not} part of this discussion:
\begin{itemize}
    \item classical ML of quantum data
    \item quantum learning theory, the connection between learning theory and
    quantum information
    \item research on similarities between learning and physical systems
\end{itemize}

\begin{figure}[ht!]
     \centering
    \includegraphics[scale=0.9]{figz/QML_approaches}
     \caption{}
  \label{fig:algvsdata} \end{figure}
\clearpage

\subsection{Basic Concepts}
Core procedure:
\begin{itemize}
    \item Encode data into quantum information processable form, such that it
    can be used on a QPU. 
    \item process information  by running algorithm
    \item result is given by measurement of final qbit states
\end{itemize}

\subsection{Advantages}
- quantum feature maps
Furthermore, to eventually obtain any quantum advantage, 
one should search from the set of classically intractable feature maps

\subsection{Supervised Learning}
\subsubsection{Classification}
The most prominent methods are quantum kernel estimates and variational quantum
classifiers.

\subsection{Unsupervised Learning}
\dots

\subsection{Taxonomy}

\paragraph{Quantum State Amplitudes} can be used to compactly encode exponential
number of information. A state with $n$ qbits has $2^n$ complex amplitudes,
allowing to design algorithms, which grow polynomially in $n$, which corresponds
to logarithmic growth in the number of amplitudes. Many of these are based on
the \textbf{HHL-algorithm}, a q-alg for solving linear systems of equations. In
a nutshell, this allows for matrix inversion which grows logarithmically as
opposed to quadratically, the most efficient known classical method. The
inversion of matrices can be used for any machine learning task relying on
solving a system of linear equations, like the least-squares versions of
\textbf{SVMs, linear regressors, Gaussian processes}. Challenges: State
preparation.

\paragraph{Grover Search} is a quantum search algorithm, which provides a
quadratic speed up of unstructured search problems. It's generalisation is
amplitude amplification. These can be used to speed up ML algorithms which are
based on such unstructured searches, eg. k-medians and k-nearest neighbour
algorithms, quantum perceptron.

\paragraph{Reinforcement Learning} depends on an agent which interacts with the
environment and receives rewards depending on its actions. In the q-realm, this
agent either possess quantum processing capabilities or probes the (classical)
environment through superposition, which both can lead to quantum speed ups. 

\paragraph{Quantum Annealing}  is a technique to determine minima of an
optimisation problem.   A problem is encoded into a Hamiltonian of a system.
This system is evolved over time; it follows the Schroedinger equation in its
time evolution. Initially, the system starts as equal superposition of all
states. Over time, it evolves into the ground state. The crux is to formulate
the problem at hand such that the ground state reflects the solution to a (ML)
problem. In particular, the form of problem must be a QUBO. 

\paragraph{Quantum Sampling} can speed up computationally hard sampling
procedures, which are at the core of many ML applications - Boltzmann Machines
as a particularly well-known example. The task is to estimate averages over
probabilistic distributions, which can be a hard task for generic probabilistic
models and is usually solved with Markov chain Monte Carlo (MCMC). Quantum
Annealing offers a different approach, here samples are drawn reluying on an
actual pysical process like quantum annealing, which naturally provides samples
following the Boltzmann distribution. 
Aside from Annealing a universal QC can also be used for probabilistic sampling:
a "thermal state" is prepared on the QPU and then measured, resulting in a BM
sample. This state is prepared using quantum enhanced Markov logic networks.

\paragraph{Quantum Neural Network} describes a class of algorithms combining
aritifical neural networks with quantum information advantages to gain speed up
and higher efficiency. Most of these are implemented as feed-forward
architectures.
Examples of QNN models are: 
\begin{itemize}
    \item Quantum perceptrons: the qbits take the place of neurons. The main
    problem here is the non-linearity required for neural activation. Quantum
    theory is linear. A practical implementation has been proposed using circuit
    based on variational quantum circuits and quantum phase estimation.
    \item Quantum Networks: theoretically generalisation of the classical neuron
    leads to general form of unitary gates. This can then be applied to a
    variety of different qbits. The learning is then still based on VQC. 
    \item Quantum Networks for algorithm design: keep the interactions between
    qbits as learnable parameter - use ML to learn the best algorithm!
    \item Quantum Associative Memory: a circuit based quantum ciomputer,
    simulating associative memory - based on Grover-like quantum search
    algorithm. The search retrieves the strate closest to the given input. Not a
    neural network in the Hopfield sense.
\end{itemize}

\paragraph{Hidden Quantum Markov Models} are the quantum equivalent to Hidden
Markov Models, a technique to learn sequential data. Density matrices are used
to express probabilities. They are tied to Bayesian Inference, which is an
interesting connection.

\paragraph{Quantum Quantum Learning} refers to the system and the computation
device being quantum states. Examples are "learning" a quantum state without
finding a classical description first. Clustering is another example allowing a
complete quantum implementation.

\subsection{Challenges}
\subsubsection{HHL-based algorithms}
\textbf{State preparation} is a crucial bottleneck for any algorithm based on
quantum state amplitude manipulation. This relates to \textbf{qRAM}.

\subsubsection{Quantum Sampling}
\textbf{State preparation} for non-annealing based sampling, "thermal state
preparation protocol". 

\subsection{Local Projects}
